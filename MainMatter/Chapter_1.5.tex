\setmainfont{Noto Serif}
\setsansfont{Noto Sans}
\setmonofont{Noto Sans Mono}
\setstretch{1.35}
\hyphenation{ма-те-ма-ти-ка вос-ста-нав-ли-вать}


\section{Правила Вигнера-Витмера}
1С. Чему равно число независимых волновых функций, которые соответствуют молекулярным термам $^1\Sigma^-$, $^3\Sigma^+$, $^3\Pi$, $^1\Phi$, $^6\Delta$?
\par
2С. Какие молекулярные термы возникают при сближении атомов углерода в~состоянии $^3P_g$ и азота в состоянии $^4S_u$?
\par
3С. Определите термы молекулы фтора, которые могут возникнуть при ее образовании в результате сведения: (1) двух атомов фтора в основном состоянии, (2)~двух атомов фтора, один из которых имеет однократно возбужденную $(s \rightarrow p)$ электронную конфигурацию, (3) ионов $\text{F}^+$ и $\text{F}^-$ в основных состояниях.
\par
4С. Определите полное количество термов, которые могут возникнуть при образовании гетероядерной молекулы из двух атомов, находящихся в состояниях $^{2S_1 + 1}L_1$ и $^{2S_2 + 1}L_2$. Для определенности считайте, что $L_1 \geq L_2$ и $S_1 \geq S_2$.
\par
\begin{wrapfigure}{r}{35mm} %this figure will be at the right
    \centering
    \vspace{0mm}
    \includegraphics[width=35mm]{images/Fig_1_5_5.png}
    \vspace{-5mm}
\end{wrapfigure}
5. Определите электронные термы, которые могут возникнуть при сближении двух атомов водорода и~двух атомов дейтерия, как это показано на рисунке, с образованием линейной молекулы $\text{H}_2 \text{D}_2$.
\par
6К. Определите термы молекулы, образующейся при сведении атомов бериллия и кислорода в основных состояниях. Может ли при этом получиться основное состояние молекулы $\text{BeO}$? Если нет, то предложите возможное возбужденное состояние одного из атомов в паре (соответствующий терм атома и~электронная конфигурация), при котором возможно образование молекулы в~основном состоянии.
\par
7К. В газовой фазе при низких температурах исследовался распад молекулы $\text{BeF}$, находящейся в электронном состоянии $^1 \Sigma ^-$ . Можно ли однозначно сказать по гетеролитическому ($\text{Be}^+$, $\text{F}^-$) или гомолитическому ($\text{Be}$, $\text{F}$) механизму идет данная реакция, если в результате проведенных экспериментов были зарегистрированы продукты распада $\text{BeF}$ только в своих основных электронных состояниях?
\par
8. Может ли при сведении атомов $\text{C}$ ($2s^2 2p^2$) и $\text{O}$ ($2s^2 2p^4$) в основных состояниях получиться молекула $\text{CO}$ в основном состоянии?
\par
9. Определить молекулярные термы, которые могут возникнуть при сближении двух атомов гелия, находящихся в нижнем по энергии возбужденном состоянии, полученным из основного под действием электромагнитного излучения.
\par
10. Определите полное количество четных термов, которые могут возникнуть при образовании гетероядерной молекулы из двух атомов, находящихся в~одинаковых состояниях $^{2S + 1}L$.
\par
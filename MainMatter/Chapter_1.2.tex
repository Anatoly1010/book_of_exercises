\setmainfont{Noto Serif}
\setsansfont{Noto Sans}
\setmonofont{Noto Sans Mono}
\setstretch{1.35}
\hyphenation{ма-те-ма-ти-ка вос-ста-нав-ли-вать дан-ный}


\section{Термы многоэлектронного атома}

1С. Определите все термы и мультиплеты электронной конфигурации $p^3$ в~пределе слабого по сравнению с межэлектронным отталкиванием взаимодействием между спиновым и орбитальным моментом.
\par
2С. Определите основной терм и основной мультиплет для электронной конфигурации $6s^24f^3$ в пределе слабого по сравнению с межэлектронным отталкиванием спин-орбитального взаимодействия.
\par
3С. Определите все термы и мультиплеты электронной конфигурации $2p^23p^3$ в~пределе слабого по сравнению с межэлектронным отталкиванием спин-орби-тального взаимодействия.
\par
4С. Определите основной терм электронной конфигурации $l^n$ в пределе слабого по сравнению с межэлектронным отталкиванием спин-орбитального взаимодействия.
\par
5С. Для электронной конфигурации $l^{4l}$ в предельном случае слабого по сравнению с межэлектронным отталкиванием спин-орбитального взаимодействия определите основной терм, общее число термов, а также число термов определенной мультиплетности.
\par
6С. Постройте корреляционную диаграмму между предельными случаями слабого и сильного спин-орбитального взаимодействия для электронной конфигурации $p^5s^1$.
\par
7С. Определите, какому терму и какой электронной конфигурации атома принадлежит волновая функция, имеющая вид:
\begin{equation*}
\begin{aligned}
\psi &=\frac{1}{\sqrt{6}}[\sqrt{2}(\textit{Y}_{21}\textit{Y}_{11}-\textit{Y}_{11}\textit{Y}_{21})+(\textit{Y}_{22}\textit{Y}_{10}-\textit{Y}_{10}\textit{Y}_{22})]\cdot\xi_S, \hspace{28mm}
\end{aligned}
\end{equation*}
где $\xi_S$ представляет собой некоторую спиновую частью волновой функции.
\par
8С. Определите все термы электронной конфигурации $p^3$ в~пределе сильного, по сравнению с межэлектронным отталкиванием, спин-орбитального взаимодействия.
\par
9С. В пределе LS связи определите все термы электронной конфигурации $h^2$.
\par
10К.  Пять первых уровней основного терма атома железа $[\text{Ar}]3d^64s^2$ имеют следующие значения относительной энергии (см$^{-1}$): 0; 415,9; 704,0; 888,1; 978,1. Определите величину константы спин-орбитального взаимодействия.
\par
11. В пределе LS связи определите все функции основного терма электронной конфигурации $p^3$.
\par
12. Определите возможную электронную конфигурацию атома, если известно, что без учета спин-орбитального взаимодействия ее основной терм $^9G$.
\par
13. Определите чему равен параметр $m$, соответствующий максимальному вырождению основного терма электронной конфигурации $l^m$.
\par
14. В пределе LS связи определите количество термов с максимальным спином для электронной конфигурации $l^3$.
\par
15К. Определите основной терм конфигурации $s^1l^n$ в случае слабого спин-орби-тального взаимодействия.
\par
16С. Сколько различных состояний с определенным значением $J$ будет возникать из электронной конфигурации $l^2$ в случае слабого и сильного спин-орби-тального взаимодействия.
\par
17К. Покажите, что средняя энергия всех мультиплетов, возникающих из терма $^{2S+1}L$, равна энергии этого терма. Для определенности считайте, что $L\ge S$. Дополнительные справочные данные:
\begin{equation*}
\begin{aligned}
&\sum_{i=0}^{n} i = \frac{1}{2}n(n+1);\quad \sum_{i=0}^{n} i^{\,2} = \frac{1}{6}n(n+1)(2n+1);\quad \sum_{i=0}^{n} i^{\,3} = \frac{1}{4}n^2(n+1)^2;\hspace{11mm}
\\[6pt]
 &\sum_{i=k}^{n+k} i = \frac{1}{2}(2k+n)(n+1);\quad \sum_{i=k}^{n+k} i^{\,2} = \frac{1}{6}(n+1)(6k^2+n+6kn+2n^2);\,
 \\[6pt]
&\sum_{i=k}^{n+k} i^{\,3} = \frac{1}{4}(n+1)(2k+n)(2k^2+n+2kn+n^2).
\end{aligned}
\end{equation*}
\par
18К. Между предельными случаями LS и jj связей возможны различные промежуточные случаи, которые реализуются в основном для электронных конфигураций, возникающих при возбуждении одного электрона из замкнутой электронной оболочки. Одним из таких примеров является JK связь. В этом случае обозначение термов имеет вид: $^{2s_2+1}[K]_J$ , где $\vec{l}_1+\vec{s}_1=\vec{j}_1$, $\vec{j}_1+\vec{l}_2=\vec{K}$, $\vec{K}+\vec{s}_2=\vec{J}$, a $\vec{s}_2$, $\vec{l}_2$ – моменты внешнего электрона, возбуждаемого из замкнутой оболочки. Определите все термы иона $\text{Kr}^{+26}$ в первом возбужденном состоянии c электронной конфигурацией $[$Be$]2p^5 3d^1$.
\par
19. Найти нормировочный множитель $A$ для радиальных слейтеровских функций $R_n(r)=A\left( \sfrac{r}{a_0} \right) ^{n^*-1}\exp[-\xi r/ a_0]$, где $n^*$ – эффективное главное квантовое число, $\xi=\frac{Z-s}{n^*}$, $Z$ – заряд ядра, $s$ – константа экранирования, $a_0$ – радиус Бора. Найти радиус, при котором плотность вероятности данных функций максимальна. Пользуясь полученным выражением, оценить размер атома железа.
\par

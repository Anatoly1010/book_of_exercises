\setmainfont{Noto Serif}
\setsansfont{Noto Sans}
\setmonofont{Noto Sans Mono}
\setstretch{1.35}
\footnotesize

\section{Угловой момент, сложение моментов}

1. Найти среднее значение потенциальной энергии в $1s$ состоянии атома водорода. Как полученная величина соотносится с полной и кинетической энергией? 
\par
2. Определите разницу в энергиях $1s$ орбиталей атома водорода и дейтерия.
\par
3. Сверхтонкое взаимодействие между спином электрона и спином ядра задается гамильтонианом $\widehat H = a (\vec S, \vec I\,)$, где $a$ – константа сверхтонкого взаимодействия, $\vec S$ и $\vec I$ – векторы, составленные из операторов электронного и ядерного спина. Определите на сколько уровней расщепится основное состояние атома водорода при учете этого взаимодействия, а также в явном виде выпишите полные волновые функции получившихся состояний.
\par
4. К атому водорода внезапно прилетает протон и прилипает к ядру. Определите чему равна вероятность остаться в основном состоянии?
\par
5С. Постройте полярные диаграммы вещественных сферических гармоник $\textit{Y}_{00}$, $\textit{Y}_{1m}$, $\textit{Y}_{2m}$. Объясните физический смысл полученных картинок.
\par
6С. Определить число спиновых функций симметричных и антисимметричных относительно перестановок для системы, состоящей из двух слабовзаимодействующих частиц со спином $S$. Рассмотрите случаи целого и полуцелого $S$.\par
7С. Постройте связь волновых функций мультипликативного базиса и базиса полного момента для системы двух моментов $L_1 = L_2 = 1$. Какова вероятность обнаружить каждое из значений полного момента, если система находится в~состоянии $\psi = \ket{1\,0}\ket{1\,0}.$
\par
8С. Сколько различных полных моментов существует в системе двух слабовзаимодействующих моментов $L_1 > L_2$?
\par
9С. Система из $N$ электронов, где $N$ – четное число, имеет $K$ возможных значений полного спина. Установите связь между $N$ и $K$. Сколько раз в данной системе встретится $K = 0$? Что изменится, если $N$ будет нечетным числом?
\par
10. Постройте матрицу оператора понижения $\widehat{S}_{-}$ для частицы со спином 5/2.
\par
11. Записать оператор скалярного произведения спинов двух слабо взаимодействующих частиц со спинами $S_1$ и $S_2$. Чему равно среднее значение этого оператора в триплетном и синглетном состояниях системы из двух спинов 1/2?
\par
12. Определите кратность вырождения электронных уровней в атоме водорода в состоянии $\ket{n\,l\,m} $.
\par
13К. Система из двух слабовзаимодействующих моментов $S_1$, $S_2$ находится в~состоянии с волновой функцией $\ket{S_1\,S_1-1}\ket{S_2\,S_2}$. Определите вероятности обнаружения различных значений полного момента $S$, а также среднее значение квадрата полного момента $\braket{S^2}$.
\par
14С. Для системы из трех слабовзаимодействующих частиц со спинами $S + n$, $S$~и~$S - n$ определите общее число возможных значений полного спина. Считайте для определенности, что $n > 0$, a $S \geq n$.
\par
15. Система из $2N$ электронов находится в состоянии, описываемой волновой функцией: $\alpha_1\beta_2\beta_3 \ldots \beta_{2N}$. Определите вероятность различных значений полного спина $S$ в этом состоянии.
\par
16. Постройте связь волновых функций мультипликативного базиса и базиса полного момента для системы трех электронов.
\par
17С. Определите количество состояний с определенными проекциями полного момента для системы из $N$ электронов. Как изменится ответ, если рассмотреть систему частиц с моментом 1.
\par
18К. Запишите полные волновые функции системы, состоящей из двух слабовзаимодействующих частиц с координатными функциями $\phi_1$ и $\phi_2$, для случаев: (1) неразличимых частиц со спином 1/2, (2) различимых частиц со спином 1/2, (3) неразличимых частиц со спином 0. В какой ситуации в реальной системе возможна реализация случая (2)?
\par
19. Определите чему равны следующие матричные элементы: $\braket{\textit{Y}_{22}|\widehat L^4_+|\textit{Y}_{2-2}}$, $\braket{\textit{Y}_{21}|3\widehat L_{z}^2- \widehat L^2|\textit{Y}_{21}}$, $\braket{\textit{Y}_{21}\textit{Y}_{20}|3 \widehat L_{z_1}^2- \widehat L_1^2+3 \widehat L_{z_2}^2- \widehat L_2^2|\textit{Y}_{21}\textit{Y}_{20}}$.
\par
20. Система из $N$ электронов находится в состоянии, описываемой волновой функцией $\psi=\alpha_1\alpha_2 \ldots \alpha_n\beta_{n+1}\beta_{n+2} \ldots \beta_{N}$. Определите среднее значение квадрата суммарного спина системы.
\par

\setmainfont{Noto Serif}
\setsansfont{Noto Sans}
\setmonofont{Noto Sans Mono}
\setstretch{1.35}
\hyphenation{ма-те-ма-ти-ка вос-ста-нав-ли-вать}


\section{Дипольные переходы между термами}

1К. Рассчитайте величину электрического дипольного момента перехода между атомными орбиталями $2s$ и~$2p_z$ в~атоме водорода.
\par
2К. Возбужденное состояние атома кальция имеет электронную конфигурацию $1s^22s^22p^63s^23p^63d^14f^1$. Определите все мультиплеты этой конфигурации. Укажите какой из мультиплетов является основным. Атом кальция находится в~состоянии $^3F_2$, порождаемом некоторой другой возбужденной конфигурацией. В какие мультиплеты из найденных разрешен электрический дипольный переход? Какой четностью должна обладать данная возбужденная конфигурация?
\par
3К. Для электронной конфигурации $6s^2 4f^4$ определите: (1) полное количество термов, (2) основной терм, считая спин-орбитальное взаимодействие слабым, (3) одну возбужденную конфигурацию, в которую возможен электрический дипольный переход из исходной электронной конфигурации.
\par
4К. В атоме углерода исследуют электрические дипольные переходы между следующими электронными конфигурациями:
\begin{equation*}
\begin{aligned}
\relax [\text{He}]2s^2 2p^2 \rightarrow \{[\text{He}]2s^1 2p^2 3s^1, [\text{He}]2s^1 2p^2 3p^1, [\text{He}]2s^2 2p^1 3p^1\}.\hspace{19mm}
\end{aligned}
\end{equation*}
Пренебрегая спин-орбитальным взаимодействием, определите сколько всего возможно переходов между данными конфигурациями. При учете спин-орбитального взаимодействия тонкая структура одного из переходов между триплетными термами состоит из 5 линий. Определите каким именно термам соответствует этот переход. Эффективная константа спин-орбитального взаимодействия одинакова во всех термах.
\par
5К. В спектре поглощения атома кальция среди прочих наблюдаются три серии линий, обозначенных A, B, C и соответствующих длинам волн (нм): A:~\{428,30; 428,94; 429,90; 430,25; 430,77; 431,87\}, B: \{394,89; 395,71; 397,37\}, C: \{585,74; 551,30\}. Какие из указанных серий относятся к переходам между мультиплетами электронных конфигураций (1): $[\text{Ar}]4s^14p^1 \rightarrow [\text{Ar}]4p^2$, а какие к переходам между мультиплетами электронных конфигураций (2): $[\text{Ar}]4s^14p^1$ $\rightarrow$ $[\text{Ar}]4p^16s^1$.
\par
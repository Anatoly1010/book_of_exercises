\setmainfont{Noto Serif}
\setsansfont{Noto Sans}
\setmonofont{Noto Sans Mono}
\setstretch{1.35}

\newpage
\section[Описание электронного строения  атомов и молекул]{\texorpdfstring{Описание электронного строения атомов и молекул}{Описание электронного строения  атомов и молекул}}
\textbf{1. Атомы водорода и гелия}\par
1.1П. Атомные орбитали. Визуализация сферических гармоник.\par
1.2. Атом гелия. Расчет энергии основного состояния по теории возмущений.\par
1.3П. Атом гелия. Расчет энергии основного состояния с помошью вариационного метода.\par
1.4. Спин-орбитальное взаимодействие. Природа, гамильтониан, константа.\par
1.5П. Тождественность частиц. Принцип Паули. Детерминант Слейтера.\par
\textbf{2. Многоэлектронные атомы}\par
2.1П. Классификация термов в пределе Рассел-Саундерса. Обозначения термов. Термы электронной конфигурации $p^2$.\par
2.2. Правила Гунда\par
2.3П. Классификация термов в пределе jj-связи. Обозначения термов. Термы электронной конфигурации $p^2$.\par
2.4. Метод Хартри-Фока. Эффективный потенциал.\par
\textbf{3. Переходы между термами}\par
3.1. Правила отбора для электрических дипольных переходов в водородоподобном атоме.\par
3.2П. Правила отбора для электрических дипольных переходов в многоэлектронном атоме.\par
\textbf{4. Многоэлектронный атом во внешних полях}\par
4.1П. Полный магнитный момент атома.\par
4.2П. Многоэлектронный атом в слабом магнитном поле. Эффект Зеемана.\par
4.3П. Многоэлектронный атом в сильном магнитном поле. Эффект Пашена-Бака.\par
4.4. Многоэлектронный атом в электрическом поле. Линейный эффект Штарка.\par
\textbf{5. Электронное строение двухатомных молекул}\par
5.1П. Классификация термов двухатомных молекул.\par
5.2. Сближение атомов с образованием молекулы. Правила Вигнера-Витмера.\par
5.3П. Молекула водорода и ее молекулярные термы.\par
5.4П. Молекула кислорода и ее молекулярные термы.\par
5.5. Правила отбора для электрических дипольных переходов в двухатомных молекулах\par
\textbf{6. Электронное строение многоатомных молекул}\par
6.1П. Приближение Борна-Оппенгеймера.\par
6.2П. Метод Рэлея-Ритца.\par
6.3П. Молекула водорода. Метод Хюккеля.\par
\textbf{7. Применение теории групп для описания электронного строения молекул}\par
7.1. Правила отбора. Оценка интегралов по теории групп.\par
7.2П. Описание $\pi$-сопряженных систем с помощью метода Хюккеля.\par
7.3. Описание электронного строения молекулы воды на языке теории групп.\par
\textbf{8. Специальные хюккелевские системы}\par
8.1П. Альтернантные системы. Теорема парности. Электронное строение аллильного радикала.\par
8.2П. Графическое представление решений для циклических систем. Электронное строение циклобутадиена.\par
8.3. Линейные системы. Электронное строение бутадиена.\par
\textbf{9. Применение теории возмущений для описания электронного строения молекул}\par
9.1П. Введение гетероатома в $\pi$-систему. Электронное строение пиридина.\par
9.2П. Введение индуктивных заместителей в $\pi$-систему. Электронное строение толуола.\par
9.3П. Направление преимущественного замещения в реакциях $S_NAr$ и $S_EAr$.\par
\textbf{10. Разное}\par
10.1. Непересечение состояний с одинаковой симметрией.\par
10.2. Метод Томаса-Ферми.\par
\textbf{11. Примеры дополнительных вопросов}\par
11.1. Может ли существовать нечетный $S$ терм?\par
11.2. Описать электронное строение молекулы N$_2$ используя теорию МО ЛКАО.\par
11.3. Используя теорию групп опишите электронное строение циклопропенильного радикала.\par
11.4. В какие состояния разрешены электрические дипольные переходы из~состояния $^1\Sigma^+$ в группе симметрии $C_{\infty v}$?\par
11.5. С помощью метода МО ЛКАО покажите, что $\text{He}_2$ неустойчивая, а $\text{He}_2^+$ устойчивая молекула.\par
\newpage

\section[Электронное строение координационных соединений.\\Спектроскопические методы исследования вещества]{\texorpdfstring{Электронное строение координационных соединений.\\Спектроскопические методы исследования вещества}{Электронное строение координационных соединений. Спектроскопические методы исследования вещества}}
\textbf{1. Правила Вудворда-Хоффмана}\par
1.1. Использование представлений симметрии при рассмотрении реакционной способности. Типы реакций. Влияние на стереохимию продукта.\par
1.2П. Описание электроциклических реакций. Кон- и дисротаторные пути протекания реакций.\par
1.3. Описание реакций циклоприсоединения на примере $[2+2]$ циклоприсоединения. Супра-/супра- и супра-/антара- пути протекания реакций.\par
\textbf{2. Строение и спектроскопия координационных соединений}\par
2.1П. Теория кристаллического поля. Гамильтониан, предельные случаи, основные приближения. Разложение Лапласа.\par
2.2. Симметрия сферических гармоник.\par
2.3. Описание электронной конфигурации $d_1$ в среднем поле октаэдрической симметрии. В частности, будет требоваться получить потенциал кристаллического поля или рассчитать расщепление по заданному потенциалу.\par
2.4. Теория поля лигандов. Учет $\sigma$ и $\pi$ орбиталей лигандов.\par
2.5П. Оптические переходы в координационных соединениях. Природа $dd$-пере-ходов.\par
2.6. Диаграммы Танабе-Сугано. Спектрохимический ряд лигандов.\par
\textbf{3. Электронная спектроскопия}\par
3.1П. Процессы, протекающие с молекулой после возбуждения.\par
3.2. Природа интеркомбинационной конверсии.\par
3.3. Принцип Франка-Кондона.\par
3.4П. Анизотропия люминесценции на примере поляризованного по оси $z$ возбуждающего излучения и равномерно распределенного замороженного образца.\par
3.5П. Уравнение Штерна-Фольмера. Время жизни возбужденного состояния.\par
3.6П. Тушение люминесценции. Обмен электронами. Триплет-триплетная аннигиляция.\par
3.7П. Тушение люминесценции. Обмен энергией.\par
3.8. Принцип работы лазера. Трехуровневые и четырехуровневые схемы создания инверсной населенности.\par
\textbf{4. ИК и КР спектроскопии}\par
4.1П. Колебания $N$-атомной молекулы. Гамильтониан, нормальные координаты, гармоническое приближение.\par
4.2. Симметрия колебаний. Характеры всех смещений ядер на примере молекулы воды. Аксиальные и полярные векторы. Классификация колебаний. Вывод строчки характеров не требуется.\par
4.3П. Правила отбора. Фундаментальные частоты, составные частоты, обертона. Примеры характеристических колебаний и их частоты.\par
4.4. Интерферометр Майкельсона. Устройство ИК спектрометра.\par
4.5П. Комбинационное рассеяние света. Рамановская спектроскопия. Правила отбора.\par
4.6. Сравнение разрешенных симметрий колебаний в рамановской и ИК спектроскопии на примере конкретной молекулы.\par
\textbf{5. Эффект Яна-Теллера}\par
5.1. Адиабатическое приближение.\par
5.2. Теорема Яна-Теллера.\par
5.3. Псевдоэффект Яна-Теллера. Динамический и статический эффекты.\par
5.4П. Тетрагональное искажение октаэдрического комплекса с конфигурацией металла $d^9$. Симметрия колебания, приводящего к искажению комплекса.\par
\textbf{6. Вращательная спектроскопия и ее производные}\par
6.1. Классификация молекул по компонентам момента инерции.\par
6.2П. Правила отбора вращательных переходов.\par
6.3. Вращательно-колебательная спектроскопия. P, Q, R-ветви.\par
6.4П. Рамановская вращательная спектроскопия. Ядерная спиновая статистика на примере молекулы водорода. Практические применения пара-водорода.\par
6.5. Оператор перестановки ядер. Классификация вращательных уровней.\par
\textbf{7. ЭПР спектроскопия}\par
7.1П. Основные понятия магнитного резонанса. Резонансные условия в ЭПР и~ЯМР. Численные оценки.\par
7.2П. ЭПР в жидкости. Характерные взаимодействия. Природа контактного сверхтонкого взаимодействия. Правила отбора.\par
7.3П. Спектры ЭПР радикалов с одним ядром: $S = I = 1/2$; $S = 1/2$, $I = 1$. Гамильтониан, структура энергетических уровней.\par
7.4П. Природа контактного СТВ в органических радикалах. Косвенное взаимодействие. Соотношение Мак-Коннела.\par
7.5П. Природа контактного СТВ в органических радикалах. Сверхсопряжение.\par
\textbf{8. Химический обмен в магнитом резонансе}\par
8.1. Ларморовская прецессия спина в магнитном поле.\par
8.2. Проявление обмена в спектрах ЭПР. Медленный и быстрый обмен, описание трансформации спектра на конкретном примере.\par
8.3. Обмен по многим положениям. Реакции вырожденного электронного обмена.\par
8.4. Поправка к энергии ЭПР перехода во втором порядке теории возмущений. Спектр $^{\boldsymbol{\cdot}}\text{CF}_3$ радикала с ее учетом.\par
\textbf{9. ЯМР спектроскопия}\par
9.1П. Гамильтониан и химический сдвиг. Природа химического сдвига. Демонстрация эффекта электроотрицательности соседних групп.\par
9.2П. Спектр ЯМР АХ системы. Как будет выглядеть спектр ЯМР $\text{A}_{\text{n}}\text{X}_{\text{m}}$ системы?\par
9.3. Химические сдвиги $^1$Н ЯМР спектроскопии. Основные факторы. Магнитная анизотропия соседних групп на примере ароматических соединений.\par
9.4. Спектр ЯМР АB системы.\par
9.5П. Особенности спектроскопии $^{13}$С ЯМР. Пример спектра $^{13}$С ЯМР соединения на выбор.\par
\textbf{10. Примеры дополнительных вопросов}\par
10.1. Гомоядерные двухатомные молекулы с нулевым ядерным спином могут иметь либо четные, либо нечетные вращательные уровни. Можно ли установить четность, исходя из вращательных спектров комбинационного рассеяния?\par
10.2. Спектр $^{1}$Н ЯМР некой молекулы, зарегистрированный в спектрометре с~частотой 400 МГц, состоит из двух линий с химическим сдвигом 0 м. д. и~10~м.~д. Чему равна протяженность спектра в герцах? На что влияет увеличение частоты ЯМР спектрометра?\par
10.3. Спектры поглощения комплексов $[\text{PtBr}_4]^{2-}$ и $[\text{PtCl}_4]^{2-}$ похожи. Однако полоса переноса заряда от лиганда к центральному иону в $[\text{PtBr}_4]^{2-}$ сдвинута в красную область, и соответствующая энергия перехода равна 36000 см$^{-1}$, в то время как для $[\text{PtCl}_4]^{2-}$ эта величина равна 44000 см$^{-1}$. Какова природа наблюдаемого сдвига?\par
10.4. Расщепление на $\beta$-протонах в спектре ЭПР этильного радикала составляет 26,87 Гс. Чему равна спиновая плотность на $1s$ орбитали $\beta$-атома водорода.\par
10.5. Определить число четных и нечетных колебаний для линейной $N$-атомной молекулы с центром инверсии.\par
10.6. Может ли в реакции $[2+2]$ циклоприсоединения двух этиленов, протекающей фотохимически в супра-/супра- топологии, получиться электронно-возбужденная молекула циклобутана?\par
10.7. Почему во вращательно-колебательном спектре молекулы NO есть Q-ветвь, а в аналогичном спектре HCl ее нет. Определите основные термы данных молекул.\par



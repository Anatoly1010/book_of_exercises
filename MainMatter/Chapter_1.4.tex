\setmainfont{Noto Serif}
\setsansfont{Noto Sans}
\setmonofont{Noto Sans Mono}
\setstretch{1.35}
\hyphenation{ма-те-ма-ти-ка вос-ста-нав-ли-вать}


\section{Многоэлектронный атом во внешних полях}

1С. Определите расщепление самой длинноволновой линии поглощения иона $\text{Ti}^{3+}$ в слабом магнитном поле. Рассмотрите случаи комнатной температуры и~температуры, близкой к абсолютному нулю.
\par
2С. Найти один мультиплет, $g$-фактор которого равен $-10$.
\par
3. Определить на сколько линий расщепляется спектральная линия, соответствующая переходу $^{2S+1}L_J$ $\rightarrow$ $^{2S+1}[L-1]_{J-1}$, в слабом магнитном поле при использовании неполяризованного света.
\par
4К. Тонкая структура линии, соответствующей в некотором атоме электронному переходу между термами $^3P \,\rightarrow\,^3D$, в сильном магнитном поле имеет вид~(см$^{-1}$): 21977,9; 21978,02; 21978,14; 21978,96; 21979,78; 21979,9; 21980,02. Определите эффективную константу спин-орбитального взаимодействия, считая, что она в указанных термах одинакова. Чему равна индукция магнитного поля, в котором был зарегистрирован описанный спектр? Магнетон бора равен 14~ГГц$\cdot$Тл$^{-1}$, скорость света равна 3$\cdot$10$^8$ м$\cdot$с$^{-1}$.
\par
5С. Найти расщепление самой длинноволновой линии атома бора в сильном магнитном поле.
\par
6С. На сколько уровней с различной энергией расщепляется основной терм атома с электронной конфигурации $l^l$ в сильном магнитном поле? Какое максимальное вырождение соответствует этим уровням и сколько будет уровней с найденным максимальным вырождением?
\par
7. Для возбужденного состояния атома водорода с $n = 2$ реализуются мультиплеты $^2S_{1/2}$, $^2P_{1/2}$, $^2P_{3/2}$. Первые два мультиплета имеют одинаковую энергию. Чему будет равна разница в энергии между мультиплетами $^2S_{1/2}$, $^2P_{1/2}$ после включения постоянного электрического поля напряженностью $E$.
\par
8. Спиновая мультиплетность основного терма электронной конфигурации, заполненной более чем наполовину, равна $S$. В этом состоянии имеется $k$ пар спаренных электронов. Определите $g$-фактор наименьшего по энергии мультиплета, возникающего из этого терма.
\par
9. Определить число линий и расстояние между ними при расщеплении в~слабом магнитном поле второй линии серии Бальмера атома водорода при поляризации, совпадающей с направлением внешнего магнитного поля.
\par
10. Определить количество уровней с различной энергией, на которое расщепится уровень атома водорода с главным квантовым числом $n = 100$ в~постоянном электрическом поле напряженностью $E$.
\par
11С. Определить расщепление электронной конфигурации $p^2$ в очень сильном магнитном поле, энергия взаимодействия с которым больше, чем энергия межэлектронного отталкивания, но меньше, чем энергия кулоновского взаимодействия электронов с ядром.
\par
12. Определите $g$-фактор основного терма электронной конфигурации $2s^12p^1$ в~случае jj связи.
\par
13С. Качественно нарисовать спектр поглощения атома водорода, помещенного во внешнее электрическое поле напряженностью $E$, в районе второй линии Лаймана. Указать поляризацию компонент спектра.
\par
14К. Основной терм электронной конфигурации $l^4$ вырожден 65 раз. Определите, что это за терм и чему равен $l$, считая спин-орбитальное взаимодействие слабым. На сколько уровней и с каким расстоянием между ними расщепится основной мультиплет найденного терма в слабом, по сравнению со спин-ор-битальным взаимодействием, магнитном поле?
\par
15. Рассчитайте $g$-факторы мультиплетов с наименьшей и наибольшей энергией, возникающих из основного терма электронной конфигурации $[2k]^{4k}$, где $k>0$ в пределе слабого по сравнению с межэлектронным отталкиванием взаимодействием между спиновым и орбитальным моментом.
\par

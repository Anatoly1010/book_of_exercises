\setmainfont{Noto Serif}
\setsansfont{Noto Sans}
\setmonofont{Noto Sans Mono}
\setstretch{1.15}

%\phantomsection
\section{Вид некоторых сферических гармоник}
%\addcontentsline{toc}{section}{Вид некоторых сферических гармоник}
%\markboth{Вид некоторых сферических гармоник}{Вид некоторых сферических гармоник}
\rule{\textwidth}{0.4pt}
\[ \textit{Y}_{00}=\sqrt{\frac{1}{4\pi}};\]
\rule{\textwidth}{0.4pt}
\[\textit{Y}_{10}=\frac{1}{2}\sqrt{\frac{3}{\pi}}\cos {\theta}; \quad \textit{Y}_{11}=-\frac{1}{2}\sqrt{\frac{3}{2\pi}}\sin{\theta} \, e^{\,i\phi};\quad \textit{Y}_{1-1}=\frac{1}{2}\sqrt{\frac{3}{2\pi}}\sin{\theta} \, e^{-i\phi}; \]
\rule{\textwidth}{0.4pt}
\[ \textit{Y}_{20}=\frac{1}{4}\sqrt{\frac{5}{\pi}}\left( 3 \cos^2{\theta}-1 \right);\] 
\[\textit{Y}_{21}=-\frac{1}{2}\sqrt{\frac{15}{2\pi}}\cos{\theta} \sin{\theta}\, e^{\,i\phi};\quad \textit{Y}_{2-1}=\frac{1}{2}\sqrt{\frac{15}{2\pi}}\cos{\theta} \sin{\theta}\, e^{-i\phi};
\]\hspace{\parindent}
\[ \textit{Y}_{22}=\frac{1}{4}\sqrt{\frac{15}{2\pi}} \sin^2 {\theta} \, e^{\,2i\phi}; \quad
\textit{Y}_{2-2}=\frac{1}{4}\sqrt{\frac{15}{2\pi}} \sin^2 {\theta} \, e^{-2i\phi};
\]
\rule{\textwidth}{0.4pt}
\[ \textit{Y}_{30}=\frac{1}{4}\sqrt{\frac{7}{\pi}}\left(5 \cos^3{\theta}-3\cos{\theta} \right);\ \]
\[\textit{Y}_{31}=-\frac{1}{8}\sqrt{\frac{21}{\pi}}\left(5\cos^2{\theta}-1 \right)\sin{\theta}\, e^{\,i\phi};\quad \textit{Y}_{3-1}=\frac{1}{8}\sqrt{\frac{21}{\pi}}\left( 5\cos^2{\theta}-1 \right)\sin{\theta}\, e^{-i\phi};
\]\hspace{\parindent}
\[\textit{Y}_{32}=\frac{1}{4}\sqrt{\frac{105}{2\pi}}\cos{\theta} \sin^2{\theta}\, e^{\,2i\phi};\quad \textit{Y}_{3-2}=\frac{1}{4}\sqrt{\frac{105}{2\pi}}\cos{\theta} \sin^2{\theta}\, e^{-2i\phi};
\]\hspace{\parindent}
\[ \textit{Y}_{33}=-\frac{1}{8}\sqrt{\frac{35}{\pi}} \sin^3 {\theta} \, e^{\,3i\phi}; \quad
\textit{Y}_{3-3}=\frac{1}{8}\sqrt{\frac{35}{\pi}} \sin^3 {\theta} \, e^{-3i\phi};
\]
\rule{\textwidth}{0.4pt}
\[ \textit{Y}_{40}=\frac{3}{16}\sqrt{\frac{1}{\pi}}\left(35 \cos^4{\theta}-30\cos^2{\theta}+3 \right);\ \]
\[\textit{Y}_{41}=-\frac{3}{8}\sqrt{\frac{5}{\pi}} \left(7\cos^3{\theta}-3\cos{\theta}\right)\sin{\theta}\, e^{\,i\phi};\; \textit{Y}_{4-1}=\frac{3}{8}\sqrt{\frac{5}{\pi}}\left(7\cos^3{\theta}-3\cos{\theta}\right)\sin{\theta}\, e^{-i\phi};
\]\hspace{\parindent}
\[\textit{Y}_{42}=\frac{3}{8}\sqrt{\frac{5}{2\pi}}\left(7\cos^2{\theta}-1 \right) \sin^2{\theta}\, e^{\,2i\phi};\quad \textit{Y}_{4-2}=\frac{3}{8}\sqrt{\frac{5}{2\pi}}\left(7\cos^2{\theta}-1 \right) \sin^2{\theta}\, e^{-2i\phi};
\]\hspace{\parindent}
\[\textit{Y}_{43}=-\frac{3}{8}\sqrt{\frac{35}{\pi}}\cos{\theta} \sin^3{\theta}\, e^{\,3i\phi};\quad \textit{Y}_{4-3}=\frac{3}{8}\sqrt{\frac{35}{\pi}}\cos{\theta} \sin^3{\theta}\, e^{-3i\phi};
\]\hspace{\parindent}
\[ \textit{Y}_{44}=\frac{3}{16}\sqrt{\frac{35}{2\pi}} \sin^4 {\theta} \, e^{\,4i\phi}; \quad
\textit{Y}_{4-4}=\frac{3}{16}\sqrt{\frac{35}{2\pi}} \sin^4 {\theta} \, e^{-4i\phi};
\]
\rule{\textwidth}{0.4pt}

%\phantomsection
\section{Действительный вид некоторых сферических гармоник}
%\addcontentsline{toc}{section}{Действительный вид некоторых сферических гармоник}
%\markboth{Действительный вид некоторых сферических гармоник}{Действительный вид некоторых сферических гармоник}
\rule{\textwidth}{0.4pt}
\[ s=\textit{Y}_{00}=\sqrt{\frac{1}{4\pi}};\]
\rule{\textwidth}{0.4pt}
\[ p_z=\textit{Y}_{10}=\sqrt{\frac{3}{4\pi}}\,\frac{z}{r}; \] 
\[p_x=\sqrt{\frac{1}{2}}\left(\textit{Y}_{1-1}-\textit{Y}_{11}\right)=\sqrt{\frac{3}{4\pi}}\,\frac{x}{r}; \quad p_y=i\sqrt{\frac{1}{2}}(\textit{Y}_{1-1}+\textit{Y}_{11})=\sqrt{\frac{3}{4\pi}}\,\frac{y}{r}; \]
\rule{\textwidth}{0.4pt}
\[ d_{z^2}=\textit{Y}_{20}=\frac{1}{4}\sqrt{\frac{5}{\pi}}\,\frac{3z^2-r^2}{r^2}; \] 
\[d_{xz}=\sqrt{\frac{1}{2}}(\textit{Y}_{2-1}-\textit{Y}_{21})=\frac{1}{2}\sqrt{\frac{15}{\pi}}\,\frac{xz}{r^2}; \quad d_{yz}=i\sqrt{\frac{1}{2}}(\textit{Y}_{2-1}+\textit{Y}_{21})=\frac{1}{2}\sqrt{\frac{15}{\pi}}\,\frac{yz}{r^2}; \]\hspace{\parindent}
\[d_{x^2-y^2}=\sqrt{\frac{1}{2}}(\textit{Y}_{2-2}+\textit{Y}_{22})=\frac{1}{4}\sqrt{\frac{15}{\pi}}\,\frac{x^2-y^2}{r^2}; \quad d_{xy}=i\sqrt{\frac{1}{2}}(\textit{Y}_{2-2}-\textit{Y}_{22})=\frac{1}{2}\sqrt{\frac{15}{\pi}}\,\frac{xy}{r^2}; \]
\rule{\textwidth}{0.4pt}
\[ f_{z^3}=\textit{Y}_{30}=\frac{1}{4}\sqrt{\frac{7}{\pi}}\,\frac{z(5z^2-3r^2)}{r^3}; \] 
\[f_{xz^2}=\sqrt{\frac{1}{2}}(\textit{Y}_{3-1}-\textit{Y}_{31})=\frac{1}{4}\sqrt{\frac{21}{2\pi}}\,\frac{x(5z^2-r^2)}{r^3}; \]\hspace{\parindent}
\[ f_{yz^2}=i\sqrt{\frac{1}{2}}(\textit{Y}_{3-1}+\textit{Y}_{31})=\frac{1}{4}\sqrt{\frac{21}{2\pi}}\,\frac{y(5z^2-r^2)}{r^3}; \]\hspace{\parindent}
\[f_{xyz}=i\sqrt{\frac{1}{2}}(\textit{Y}_{3-2}-\textit{Y}_{32})=\frac{1}{2}\sqrt{\frac{105}{\pi}}\,\frac{xyz}{r^3}; \] \hspace{\parindent}
\[ f_{z(x^2-y^2)}=\sqrt{\frac{1}{2}}(\textit{Y}_{3-2}+\textit{Y}_{32})=\frac{1}{4}\sqrt{\frac{105}{\pi}}\,\frac{z(x^2-y^2)}{r^3}; \] \hspace{\parindent}
\[f_{y(3x^2-y^2)}=i\sqrt{\frac{1}{2}}(\textit{Y}_{3-3}+\textit{Y}_{33})=\frac{1}{4}\sqrt{\frac{35}{2\pi}}\,\frac{y(3x^2-y^2)}{r^3}; \] \hspace{\parindent}
\[f_{x(x^2-3y^2)}=\sqrt{\frac{1}{2}}(\textit{Y}_{3-3}-\textit{Y}_{33})=\frac{1}{4}\sqrt{\frac{35}{2\pi}}\,\frac{x(x^2-3y^2)}{r^3}; \]
\rule{\textwidth}{0.4pt}

\setstretch{1.55}

%\phantomsection
\section{Некоторые таблицы характеров}
%\addcontentsline{toc}{section}{Некоторые таблицы характеров}
%\markboth{Некоторые таблицы характеров}{Некоторые таблицы характеров}
\begin{center}
\begin{tabular}{ |c|c|c|c|c|c| }
 \hline
 $C_{2v}$ & $E$ & $C_2^z$ & $\sigma_v^{xz}$ & $\sigma_v^{yz}$ & $h=4$ \\ 
 \hline
 $A_1$ & 1 & 1 & 1 & 1 & $z,\,x^2,\,y^2,\,z^2$ \\  
 \hline
 $A_2$ & 1 & 1 & $-1$ & $-1$ & $R_z,\,xy$ \\
 \hline
 $B_1$ & 1 & $-1$ & $1$ & $-1$ & $x,\,R_y,\,xz$ \\
 \hline
 $B_2$ & 1 & $-1$ & $-1$ & $1$ & $y,\,R_x,\,yz$ \\
 \hline
\end{tabular}
\end{center}

\begin{center}
\begin{tabular}{ |c|c|c|c|c| }
 \hline
 $C_{3v}$ & $E$ & $2C_3^z$ & $3\sigma_v$ & $h=6$ \\ 
 \hline
 $A_1$ & 1 & 1 & 1 & $z,\,x^2+y^2,\,z^2$ \\  
 \hline
 $A_2$ & 1 & 1 & $-1$ & $R_z$ \\
 \hline
 $E$ & 2 & $-1$ & $0$ & \begin{tabular}{@{}c@{}} $(x,y),\, (R_x,\,R_y)$ \\ $(x^2-y^2,\,xy),\,(xz,\,yz)$\end{tabular} \\
 \hline
\end{tabular}
\end{center}

\begin{center}
\begin{tabular}{ |c|c|c|c|c|c|c| }
 \hline
 $C_{4v}$ & $E$ & $2C_4^z$ & $C_2^z$ & $2\sigma_v$ & $2\sigma_d$ & $h=8$ \\ 
 \hline
 $A_1$ & 1 & 1 & 1 & 1 & 1 & $z,\,x^2+y^2,\,z^2$ \\  
 \hline
 $A_2$ & 1 & 1 & $1$ & $-1$ & $-1$ & $R_z$ \\
 \hline
 $B_1$ & 1 & $-1$ & 1 & 1 & $-1$ & $x^2-y^2$ \\  
 \hline
 $B_2$ & 1 & $-1$ & $1$ & $-1$ & $1$ & $xy$ \\
 \hline
 $E$ & 2 & $0$ & $-2$ & $0$ & $0$ & \begin{tabular}{@{}c@{}} $(x,y),\, (R_x,\,R_y)$ \\ $(xz,\,yz)$\end{tabular} \\
 \hline
\end{tabular}
\end{center}

\begin{center}
\begin{tabular}{ |c|c|c|c|c| }
 \hline
 $C_{\infty v}$ & $E$ & $2C_{\infty}$ & $\infty\sigma_v$ & $h=\infty$ \\ 
 \hline
 $\Sigma^+$ & 1 & 1 & 1 & $z,\,x^2+y^2,\,z^2$ \\  
 \hline
 $\Sigma^-$ & 1 & 1 & $-1$ & $R_z$ \\
 \hline
 $\Pi$ & 2 & $2\cos{\phi}$ & $0$ & \begin{tabular}{@{}c@{}} $(x,y),\, (R_x,\,R_y)$ \\ $(xz,\,yz)$\end{tabular} \\
 \hline
 $\Delta$ & 2 & $2\cos{2\phi}$ & $0$ & $(x^2-y^2,\,xy)$ \\
 \hline
 $\Phi$ & 2 & $2\cos{3\phi}$ & $0$ & $-$ \\
 \hline
 $\ldots$ & $\ldots$ & $\ldots$ & $\ldots$ & $-$ \\
 \hline
 $E_n$ & 2 & $2\cos{n\phi}$ & $0$ & $-$ \\
 \hline
\end{tabular}
\end{center}

\begin{center}
\begin{tabular}{ |c|c|c|c|c|c|c| }
 \hline
 $D_{2d}$ & $E$ & $2S_4$ & $C_2^z$ & $2C'_2$ & $2\sigma_d$ & $h=8$ \\ 
 \hline
 $A_1$ & 1 & 1 & 1 & 1 & 1 & $x^2+y^2,\,z^2$ \\  
 \hline
 $A_2$ & 1 & 1 & $1$ & $-1$ & $-1$ & $R_z$ \\
 \hline
 $B_1$ & 1 & $-1$ & 1 & 1 & $-1$ & $x^2-y^2$ \\  
 \hline
 $B_2$ & 1 & $-1$ & $1$ & $-1$ & $1$ & $z,\,xy$ \\
 \hline
 $E$ & 2 & $0$ & $-2$ & $0$ & $0$ & \begin{tabular}{@{}c@{}} $(x,y),\, (R_x,\,R_y)$ \\ $(xz,\,yz)$\end{tabular} \\
 \hline
\end{tabular}
\end{center}

\begin{center}
\begin{tabular}{ |c|c|c|c|c|c|c|c|c|c| }
 \hline
 $D_{2h}$ & $E$ & $C_2^z$ & $C_2^y$ & $C_2^x$ & $I$ & $\sigma_{xy}$ & $\sigma_{xz}$ & $\sigma_{yz}$ & $h=8$ \\ 
 \hline
 $A_g$ & 1 & 1 & 1 & 1 & 1 & 1 & 1 & 1 & $x^2,\,y^2,\,z^2$ \\  
 \hline
 $B_{1g}$ & 1 & 1 & $-1$ & $-1$ & $1$ & $1$ & $-1$ & $-1$ & $R_z$,\,$xy$ \\
 \hline
 $B_{2g}$ & 1 & $-1$ & $1$ & $-1$ & $1$ & $-1$ & $1$ & $-1$ & $R_y$,\,$xz$ \\
 \hline
 $B_{3g}$ & 1 & $-1$ & $-1$ & $1$ & $1$ & $-1$ & $-1$ & $1$ & $R_x$,\,$yz$ \\
 \hline
 $A_u$ & 1 & 1 & 1 & 1 & $-1$ & $-1$ & $-1$ & $-1$ & $-$ \\  
 \hline
 $B_{1u}$ & 1 & 1 & $-1$ & $-1$ & $-1$ & $-1$ & $1$ & $1$ & $z$ \\
 \hline
 $B_{2u}$ & 1 & $-1$ & $1$ & $-1$ & $-1$ & $1$ & $-1$ & $1$ & $y$ \\
 \hline
 $B_{3u}$ & 1 & $-1$ & $-1$ & $1$ & $-1$ & $1$ & $1$ & $-1$ & $x$ \\
 \hline
\end{tabular}
\end{center}

\begin{center}
\begin{tabular}{ |c|c|c|c|c|c|c|c| }
 \hline
 $D_{3h}$ & $E$ & $2C_3^z$ & $3C'_2$ & $\sigma_h^{xy}$ & $2S_{3}$ & $3\sigma_{v}$ & $h=12$ \\ 
 \hline
 $A'_1$ & 1 & 1 & 1 & 1 & 1 & 1 & $x^2+y^2,\,z^2$ \\  
 \hline
 $A'_{2}$ & 1 & 1 & $-1$ & $1$ & $1$ & $-1$ & $R_z$ \\
 \hline
 $E'$ & 2 & $-1$ & $0$ & $2$ & $-1$ & $0$ & $(x,\,y),\,(x^2-y^2,\,xy)$ \\
 \hline
 $A''_{1}$ & 1 & $1$ & $1$ & $-1$ & $-1$ & $-1$ & $-$ \\
 \hline
 $A''_2$ & 1 & 1 & $-1$ & $-1$ & $-1$ & $1$ &  $z$ \\  
 \hline
 $E''$ & 2 & $-1$ & $0$ & $-2$ & $1$ & $0$ & $(R_x,\,R_y),\,(xz,\,yz)$ \\
 \hline
\end{tabular}
\end{center}

\begin{center}
\begin{tabular}{ |c|c|c|c|c|c|c| }
 \hline
 $T_{d}$ & $E$ & $8C_3$ & $3C_2$ & $6S_4$ & $6\sigma_d$ & $h=24$ \\ 
 \hline
 $A_1$ & 1 & 1 & 1 & 1 & 1 & $r^2$ \\  
 \hline
 $A_2$ & 1 & 1 & $1$ & $-1$ & $-1$ & $-$ \\
 \hline
 $E$ & 2 & $-1$ & 2 & 0 & $0$ & $(3z^2-r^2,\,x^2-y^2)$ \\  
 \hline
 $T_1$ & 3 & $0$ & $-1$ & $1$ & $-1$ & $\vec{R}$ \\
 \hline
 $T_2$ & 3 & $0$ & $-1$ & $-1$ & $1$ & $\vec{r},\,(xy,\,xz,\,yz)$ \\
 \hline
\end{tabular}
\end{center}

\begin{center}
\begin{tabular}{ |c|c|c|c|c|c|c|c|c|c|c|c| }
 \hline
 $D_{4h}$ & $E$ & $2C_4^z$ & $C_2^z$ & $2C_2'$ & $2C_2''$ & $I$ & $2S_{4}$ & $\sigma_{h}$ & 2$\sigma_{v}$ & 2$\sigma_{d}$ & $h=16$ \\ 
 \hline
 $A_{1g}$ & 1 & 1 & 1 & 1 & 1 & 1 & 1 & 1 & 1 & 1 & $x^2+y^2,\,z^2$ \\  
 \hline
 $A_{2g}$ & 1 & 1 & $1$ & $-1$ & $-1$ & 1 & 1 & $1$ & $-1$ & $-1$ & $R_z$ \\
 \hline
 $B_{1g}$ & 1 & $-1$ & $1$ & $1$ & $-1$ & 1 & $-1$ & $1$ & $1$ & $-1$ & $x^2-y^2$ \\  
 \hline
 $B_{2g}$  & 1 & $-1$ & $1$ & $-1$ & $1$ & 1 & $-1$ & $1$ & $-1$ & $1$ & $xy$ \\
 \hline 
 $E_g$ & 2 & $0$ & $-2$ & $0$ & $0$ & 2 & $0$ & $-2$ & $0$ & $0$ & \begin{tabular}{@{}c@{}} $(R_x,\,R_y)$ \\ $(xz,\,yz)$\end{tabular} \\
 \hline
 $A_{1u}$ & 1 & $1$ & $1$ & $1$ & $1$ & $-1$ & $-1$ & $-1$ & $-1$ & $-1$ & $-$ \\
 \hline
 $A_{2u}$ & 1 & 1 & $1$ & $-1$ & $-1$ & $-1$ & $-1$ & $-1$ & $1$ & $1$ &  $z$ \\  
 \hline
 $B_{1u}$ & 1 & $-1$ & $1$ & $1$ & $-1$ & $-1$ & $1$ & $-1$ & $-1$ & $1$ & $-$ \\
 \hline
 $B_{2u}$ & 1 & $-1$ & $1$ & $-1$ & $1$ & $-1$ & $1$ & $-1$ & $1$ & $-1$ & $-$ \\  
 \hline
 $E_u$ & 2 & $0$ & $-2$ & $0$ & $0$ & $-2$ & $0$ & $2$ & $0$ & $0$ & $(x,\,y)$ \\
 \hline 
\end{tabular}
\end{center}
При наличии в группе симметрии одновременно осей симметрии $C_2'$ и $C_2''$ считать, что оси $C_2'$ направлены на атомы, а оси $C_2''$ направлены между атомами. Отражения в плоскости $\sigma_v$, $\sigma_d$ в этом случае содержат оси симметрии $C_2'$, $C_2''$, соответственно.

\begin{center}
\begin{tabular}{ |c|c|c|c|c|c|c|c| }
 \hline
 $D_{\infty h}$ & $E$ & $2C_{\infty}$ & $\infty\sigma_v$ & $I$ & $2S_\infty$ & $\infty C_2'$ & $h=\infty$ \\ 
 \hline
 $\Sigma^+_g$ & 1 & 1 & 1 & 1 & 1 & 1 & $x^2+y^2,\,z^2$ \\  
 \hline
 $\Sigma^-_g$ & 1 & 1 & $-1$ & 1 & 1 & $-1$ & $R_z$ \\
 \hline
 $\Pi_g$ & 2 & $2\cos{\phi}$ & $0$ & 2 & $-2\cos{\phi}$ & $0$ & $(R_x,\,R_y),\,(xz,\,yz)$ \\
 \hline
 $\Delta_g$ & 2 & $2\cos{2\phi}$ & $0$ & 2 & $2\cos{2\phi}$ & $0$ & $(x^2-y^2,\,xy)$ \\
 \hline
 $\Phi_g$ & 2 & $2\cos{3\phi}$ & $0$ & 2 & $-2\cos{3\phi}$ & $0$ & $-$ \\
 \hline
 $\ldots$ & $\ldots$ & $\ldots$ & $\ldots$ & $\ldots$ & $\ldots$ & $\ldots$ & $-$ \\
 \hline
 $E_{ng}$ & 2 & $2\cos{n\phi}$ & $0$ & 2 & $(-1)^n\,2\cos{n\phi}$ & $0$ & $-$ \\
 \hline
 $\Sigma^+_u$ & 1 & 1 & 1 & $-1$ & $-1$ & $-1$ & $z$ \\  
 \hline
 $\Sigma^-_u$ & 1 & 1 & $-1$ & $-1$ & $-1$ & $1$ & $-$ \\
 \hline
 $\Pi_u$ & 2 & $2\cos{\phi}$ & $0$ & $-2$ & $2\cos{\phi}$ & $0$ & $(x,\,y)$ \\
 \hline
 $\Delta_u$ & 2 & $2\cos{2\phi}$ & $0$ & $-2$ & $-2\cos{2\phi}$ & $0$ & $-$ \\
 \hline
 $\Phi_u$ & 2 & $2\cos{3\phi}$ & $0$ & $-2$ & $2\cos{3\phi}$ & $0$ & $-$ \\
 \hline
 $\ldots$ & $\ldots$ & $\ldots$ & $\ldots$ & $\ldots$ & $\ldots$ & $\ldots$ & $-$ \\
 \hline
 $E_{nu}$ & 2 & $2\cos{n\phi}$ & $0$ & $-2$ & $(-1)^{n+1}\,2\cos{n\phi}$ & $0$ & $-$ \\
 \hline
\end{tabular}
\end{center}

\begin{center}
\begin{tabular}{ |c|c|c|c|c|c|c|c|c|c|c|c| }
 \hline
 $O_{h}$ & $E$ & $8C_3$ & $6C_2$ & $6C_4^z$ & $3C_2^z$ & $I$ & $6S_{4}$ & $8S_{6}$ & 3$\sigma_{h}$ & 6$\sigma_{d}$ & $h=48$ \\ 
 \hline
 $A_{1g}$ & 1 & 1 & 1 & 1 & 1 & 1 & 1 & 1 & 1 & 1 & $r^2$ \\  
 \hline
 $A_{2g}$ & 1 & 1 & $-1$ & $-1$ & $1$ & 1 & $-1$ & $1$ & $1$ & $-1$ & $-$ \\
 \hline
 $E_{g}$ & 2 & $-1$ & $0$ & $0$ & $2$ & 2 & $0$ & $-1$ & $2$ & $0$ & \begin{tabular}{@{}c@{}} $(3z^2-r^2,$\\$x^2-y^2)$\end{tabular} \\  
 \hline
 $T_{1g}$  & 3 & $0$ & $-1$ & $1$ & $-1$ & 3 & $1$ & $0$ & $-1$ & $-1$ & $\vec{R}$ \\
 \hline 
 $T_{2g}$ & 3 & $0$ & $1$ & $-1$ & $-1$ & 3 & $-1$ & $0$ & $-1$ & $1$ &  \fontsize{7.82}{7.82}$(xz,yz,xy)$ \\
 \hline
 $A_{1u}$ & 1 & $1$ & $1$ & $1$ & $1$ & $-1$ & $-1$ & $-1$ & $-1$ & $-1$ & $-$ \\
 \hline
 $A_{2u}$ & 1 & 1 & $-1$ & $-1$ & $1$ & $-1$ & $1$ & $-1$ & $-1$ & $1$ &  $-$ \\  
 \hline
 $E_{u}$ & 2 & $-1$ & $0$ & $0$ & $2$ & $-2$ & $0$ & $1$ & $-2$ & $0$ & $-$ \\
 \hline
 $T_{1u}$ & 3 & $0$ & $-1$ & $1$ & $-1$ & $-3$ & $-1$ & $0$ & $1$ & $1$ & $\vec{r}$ \\  
 \hline
 $T_{2u}$ & 3 & $0$ & $1$ & $-1$ & $-1$ & $-3$ & $1$ & $0$ & $1$ & $-1$ & $-$ \\
 \hline 
\end{tabular}
\end{center}

%\newpage


\setstretch{1.35}

%\phantomsection
\section{Эквивалентные операторы для интегралов с \texorpdfstring{$\textit{Y}_{2m}$}{|2 m>}}
%\addcontentsline{toc}{section}{Эквивалентные операторы для интегралов с $\textit{Y}_{2m}$}
%\markboth{Эквивалентные операторы для интегралов с $\textit{Y}_{2m}$}{Эквивалентные операторы для интегралов с $\textit{Y}_{2m}$}
\rule{\textwidth}{0.4pt}
\[ \widehat L_{40} = \frac{1}{168}\sqrt{\frac{1}{\pi}}\left( 35\widehat L_z^4-30\widehat L_z^2 \widehat L^2+25\widehat L_z^2-6\widehat L^2+3 (\widehat L^2)^2 \right); \quad \widehat L_{44} =\frac{1}{168}\sqrt{\frac{35}{2\pi}} \left( \widehat L_+^4 + \widehat L_-^4  \right);\]\hspace{\parindent}
%\rule{\textwidth}{0.4pt}
\[ \widehat L_{41} = -\frac{1}{168}\sqrt{\frac{5}{\pi}} \left( (\widehat L_+ + \widehat L_- )(7\widehat L_z^3 - (3\widehat L^2 + 1)\widehat L_z) + (7\widehat L_z^3 - (3\widehat L^2 + 1)\widehat L_z)(\widehat L_+ + \widehat L_- ) \right); \]\hspace{\parindent}
%\rule{\textwidth}{0.4pt}
\[ \widehat L_{42} =\frac{1}{168}\sqrt{\frac{5}{2\pi}} \left( (\widehat L_+^2 + \widehat L_-^2 )(7\widehat L_z^2 - \widehat L^2 - 5) + (7\widehat L_z^2 - \widehat L^2  - 5)(\widehat L_+^2 + \widehat L_-^2 ) \right); \]\hspace{\parindent}
%\rule{\textwidth}{0.4pt}
\[ \widehat L_{43} =-\frac{1}{24}\sqrt{\frac{5}{7\pi}} \left( (\widehat L_+^3 + \widehat L_-^3 )\widehat L_z + \widehat L_z (\widehat L_+^3 + \widehat L_-^3 ) \right); \]
%\rule{\textwidth}{0.4pt}
%\[ \widehat L_{44} =\frac{1}{168}\sqrt{\frac{35}{2\pi}} \left( \widehat L_+^4 + \widehat L_-^4  \right);\]
\rule{\textwidth}{0.4pt}
\[ \widehat L_{20} = -\frac{1}{42}\sqrt{\frac{5}{\pi}}\left(3\widehat L_z^2-\widehat L^2\right); \quad \widehat L_{22} =-\frac{1}{14}\sqrt{\frac{5}{6\pi}} \left( \widehat L_+^2 + \widehat L_-^2  \right); \]\hspace{\parindent}
%\rule{\textwidth}{0.4pt}
\[ \widehat L_{21} = \frac{1}{14}\sqrt{\frac{5}{6\pi}}\left(\widehat L_z (\widehat L_+ + \widehat L_-)+(\widehat L_+ + \widehat L_-) \widehat L_z\right);\]
\rule{\textwidth}{0.4pt}
%\[ \widehat L_{22} =-\frac{1}{14}\sqrt{\frac{5}{6\pi}} \left( \widehat L_+^2 + \widehat L_-^2  \right); \]
%\rule{\textwidth}{0.4pt}

%\phantomsection
\section{Эквивалентные операторы для интегралов с \texorpdfstring{$\textit{Y}_{1m}$}{|1 m>}}
%\addcontentsline{toc}{section}{Эквивалентные операторы для интегралов с $\textit{Y}_{1m}$}
%\markboth{Эквивалентные операторы для интегралов с $\textit{Y}_{1m}$}{Эквивалентные операторы для интегралов с $\textit{Y}_{1m}$}
\rule{\textwidth}{0.4pt}
\[ \widehat L_{20} = -\frac{1}{2}\sqrt{\frac{1}{5\pi}}\left(3\widehat L_z^2-\widehat L^2\right); \quad \widehat L_{22} =-\frac{1}{2}\sqrt{\frac{3}{10\pi}} \left( \widehat L_+^2 + \widehat L_-^2  \right);\]\hspace{\parindent}
%\rule{\textwidth}{0.4pt}
\[ \widehat L_{21} = \frac{1}{2}\sqrt{\frac{3}{10\pi}}\left(\widehat L_z (\widehat L_+ + \widehat L_-)+(\widehat L_+ + \widehat L_-) \widehat L_z\right);\]
\rule{\textwidth}{0.4pt}
%\[ \widehat L_{22} =-\frac{1}{2}\sqrt{\frac{3}{10\pi}} \left( \widehat L_+^2 + \widehat L_-^2  \right); \]
%\rule{\textwidth}{0.4pt}

\setstretch{2.0}

%\phantomsection
\section{Симметрия нормальных колебаний}
%\addcontentsline{toc}{section}{Симметрия нормальных колебаний}
%\markboth{Симметрия нормальных колебаний}{Симметрия нормальных колебаний}
\vspace{0.4cm}

\begin{center}
\begin{tabular}{ |C{1cm}|c|C{1cm}|c|c| }
 \hline
 $\mathbf{R}$ & $\mathbf{\Gamma_Q}$  & $\mathbf{\Gamma_Q^{\text{вал}}}$ & $\mathbf{\Gamma_d}$ & $\mathbf{\Gamma}$\boldmath$_\alpha$\\[0.35ex]
 \hline\hline
 $E$ & $3N-6$ & $N_{\text{св}}$ & $3$ & $6$ \\[0.35ex]  
 \hline
 $C_n$ & $(N_c-2)(2\cos \phi+1)$ & $N_{\text{св}}$ & $2\cos\phi+1$ & $2\cos\phi\,(2\cos\phi+1)$ \\[0.35ex]
 \hline
 $I$ & $-3N_I$ & $N_{\text{св}}$ & $-3$ & $6$ \\[0.35ex]
 \hline
 $\sigma$ & $N_\sigma$ & $N_{\text{св}}$ & $1$ & $2$ \\[0.35ex]
 \hline
 $S_n$ & $N_S\,(2\cos \phi-1)$ & $N_{\text{св}}$ &$2\cos\phi-1$ & $2\cos\phi\,(2\cos\phi-1)$ \\[0.35ex]
 \hline
\end{tabular}
\end{center}

\begin{center}
\begin{tabular}{ |C{1cm}|c|c| }
 \hline
 $\mathbf{R}$ & $\mathbf{\Gamma_Q^{\text{пл}}}$ & $\mathbf{\Gamma_Q^{\text{лин}}}$ \\[0.35ex] 
 \hline\hline
 $E$ & $2N-3$ & $3N-5$ \\[0.35ex]  
 \hline
 $C_n^z$ & $2\cos \phi\,(N_c-1)-1$ & $(N-2)\,(2\cos\phi+1)+1$ \\[0.35ex]
 \hline
 $C_2'$ & $1$ & $1-N_C$ \\[0.35ex]
 \hline
 $I$ & $-2N_I+1$ & $-3N_I+1$ \\[0.35ex]
 \hline
 $\sigma_{xy}$ & $2N-3$ &  $N_\sigma +1 $\\[0.35ex]
 \hline
 $\sigma_{v},\,\sigma_{d}$ & $1$ & $N-1$\\[0.35ex]
 \hline
 $S_{n}$ & $2\cos \phi\,(N_S-1)-1$ & $N_S\,(2\cos\phi-1)+1$ \\[0.35ex]
 \hline
\end{tabular}
\end{center}

\setstretch{3}

%\phantomsection
\section{Симметрия сферических гармоник}
%\addcontentsline{toc}{section}{Симметрия сферических гармоник}
%\markboth{Симметрия сферических гармоник}{Симметрия сферических гармоник}

\begin{center}
\begin{tabular}{ |C{1cm}|c|c|c| }
 \hline
 $\mathbf{R}$ & $\mathbf{{Y}_{lm}}$ & $\mathbf{^{2S+1}L_g}$ & $\mathbf{^{2S+1}L_u}$ \\[1.5ex]
 \hline\hline
 $E$ & $2l+1$ & $2L+1$ & $2L+1$ \\[1.5ex]
 \hline
 $C_n$ & $\dfrac{\sin{\left( [l+1/2]\phi \right)}}{\sin(\phi/2)}$ & $\dfrac{\sin{\left( [L+1/2]\phi \right)}}{\sin(\phi/2)}$ & $\dfrac{\sin{\left( [L+1/2]\phi \right)}}{\sin(\phi/2)}$ \\[1.5ex]
 \hline
 $I$ & $(-1)^l\,(2l+1)$ & $2L+1$ & $-(2L+1)$ \\[1.5ex]
 \hline
 $\sigma$ & $1$ & $(-1)^L$ & $(-1)^{L+1}$ \\[1.5ex]
 \hline
 $S_n$ & $\dfrac{\cos{\left( [l+1/2]\phi \right)}}{\cos(\phi/2)}$ & $(-1)^L\,\dfrac{\cos{\left( [L+1/2]\phi \right)}}{\cos(\phi/2)}$ & $(-1)^{L+1}\,\dfrac{\cos{\left( [L+1/2]\phi \right)}}{\cos(\phi/2)}$  \\[1.5ex]
 \hline
\end{tabular}
\end{center}

\vspace{0.5cm}

\setstretch{1.0}
%\phantomsection
\section{Некоторые ядерные спины}
%\addcontentsline{toc}{section}{Некоторые ядерные спины}
%\markboth{Некоторые ядерные спины}{Некоторые ядерные спины}
\[I(^1\text{H})=1/2;\ I(^2\text{H})= 1;\ I(^{12}\text{C})=0;\ I(^{13}\text{C})=1/2;\ I(^{14}\text{N})=1;\ I(^{15}\text{N})=1/2;\ \] \vspace{0.1mm}
\[I(^{16}\text{O})=0;\ I(^{17}\text{O})=5/2;\ I(^{19}\text{F})=1/2;\ I(^{31}\text{P})=1/2;\ I(^{35}\text{Cl})=3/2;\ I(^{37}\text{Cl})=3/2; \]

\vspace{0.5cm}

\setstretch{1.0}
%\phantomsection
\section{Суммы некоторых рядов}
%\addcontentsline{toc}{section}{Суммы некоторых рядов}
%\markboth{Суммы некоторых рядов}{Суммы некоторых рядов}
%\newline
%\vspace{0.5mm}\newline
\[ \sum\limits_{i=1}^{n} \cos{i\phi} = \dfrac{\cos\left[{(n+1)\phi/2}\right]\sin[n\phi/2]}{\sin{[\phi/2]}};\quad
\sum\limits_{i=1}^{n} \sin{i\phi} = \dfrac{\sin\left[{(n+1)\phi/2}\right]\sin[n\phi/2]}{\sin{[\phi/2]}};\] \vspace{0.1mm}
\[ \sum\limits_{i=1}^{n} i = \dfrac{n(n+1)}{2}= \dfrac{a_1+a_n}{2}\,n;\quad 
 \sum\limits_{i=1}^{n} i^{\,2} = \dfrac{n(n+1)(2n+1)}{6};\quad
 \sum\limits_{i=1}^{n} a^{\,i} = \dfrac{a^{n+1}-1}{a-1};\] \vspace{1.5mm}
 \[ \sum\limits_{i=1}^{n} \cos^2{i\phi} = \dfrac{n}{2}+\dfrac{\cos\left[{(n+1)\phi}\right]\sin[n\phi]}{2\sin{\phi}};\quad
\sum\limits_{i=1}^{n} \sin^2{i\phi} = \dfrac{n}{2}-\dfrac{\cos\left[{(n+1)\phi}\right]\sin[n\phi]}{2\sin{\phi}};\] \vspace{0.1mm}

%\phantomsection
\section{Тригонометрические формулы}
%\addcontentsline{toc}{section}{Тригонометрические формулы}
%\markboth{Тригонометрические формулы}{Тригонометрические формулы}
\[ \cos2\alpha=2\cos^2{\alpha}-1=1-2\sin^2{\alpha}=\cos^2{\alpha}-\sin^2{\alpha}; \quad \sin{2\alpha}=2\sin\alpha\cos\alpha;\] \vspace{0.1mm}
\[ \sin(\alpha\,\pm\,\beta)=\sin\alpha\cos\beta \pm \cos\alpha\sin\beta; \quad \cos(\alpha\,\pm\,\beta)=\cos\alpha\cos\beta \mp \sin\alpha\sin\beta; \] \vspace{0.1mm}
\[ \sin(\alpha\,+\,\pi/2)=\cos\alpha; \quad \cos(\alpha\,+\,\pi/2)=-\sin\alpha; \] \vspace{0.1mm}
\[ \sin(\alpha\,+\,\pi)=-\sin\alpha; \quad \cos(\alpha\,+\,\pi)=-\cos\alpha; \quad \] \vspace{0.1mm}
\[ \sin\alpha\,+\,\sin\beta= 2\sin{\dfrac{\alpha+\beta}{2}}\cos{\dfrac{\alpha-\beta}{2}}; \quad \sin\alpha\,-\,\sin\beta= 2\sin{\dfrac{\alpha-\beta}{2}}\cos{\dfrac{\alpha+\beta}{2}};\] \vspace{0.1mm}
\[ \cos\alpha\,+\,\cos\beta= 2\cos{\dfrac{\alpha+\beta}{2}}\cos{\dfrac{\alpha-\beta}{2}}; \quad \cos\alpha\,-\,\cos\beta= -2\sin{\dfrac{\alpha+\beta}{2}}\sin{\dfrac{\alpha-\beta}{2}};\] \vspace{0.1mm}
\[ 1\,+\,\tg ^2\alpha=\dfrac{1}{\cos^2\alpha}; \quad 1\,+\,\ctg ^2\alpha=\dfrac{1}{\sin^2\alpha}; \] \vspace{0.1mm}
\[ \tg (\alpha\,+\,\beta)=\dfrac{\tg \alpha\,+\, \tg \beta}{1-\tg \alpha \cdot \tg \beta}; \quad \tg (\alpha\,-\,\beta)=\dfrac{\tg \alpha\,-\, \tg \beta}{1+\tg \alpha \cdot \tg \beta}; \] \vspace{0.1mm}
\[ \ctg (\alpha\,+\,\beta)=\dfrac{\ctg \alpha\cdot \ctg \beta\,-\,1}{\ctg \alpha\,+\,\ctg \beta}; \quad \ctg (\alpha\,-\,\beta)=\dfrac{\ctg \alpha\cdot \ctg \beta\,+\,1}{\ctg \alpha\,-\,\ctg \beta}; \] \vspace{0.1mm}
\[ \tg (\alpha\,+\,\pi/2)=-\ctg \alpha; \quad \ctg(\alpha\,+\,\pi/2)=-\tg \alpha; \] \vspace{0.1mm}
\[ \tg (\alpha\,+\,\pi)=\tg \alpha; \quad \ctg(\alpha\,+\,\pi)=\ctg \alpha; \] \vspace{0.1mm}
\vspace{0.1mm}

\setstretch{1.15}

%\phantomsection
\section{Вид некоторых радиальных функций}
%\addcontentsline{toc}{section}{Вид некоторых радиальных функций}
%\markboth{Вид некоторых радиальных функций}{Вид некоторых радиальных функций}
\rule{\textwidth}{0.4pt}
\[ R_{10}=2 \left( \dfrac{Z}{a_0} \right)^{\frac{3}{2}}e^{-Zr/a_0}; \]
\rule{\textwidth}{0.4pt}
\[ R_{20}=2 \left( \dfrac{Z}{2a_0} \right)^{\frac{3}{2}}\left( 1-\dfrac{Zr}{2a_0} \right)e^{-Zr/2a_0}; \quad R_{21}=\dfrac{1}{\sqrt{3}} \left( \dfrac{Z}{2a_0} \right)^{\frac{3}{2}}\left( \dfrac{Zr}{a_0} \right)e^{-Zr/2a_0};\]
\rule{\textwidth}{0.4pt}
\[ R_{30}=2 \left( \dfrac{Z}{3a_0} \right)^{\frac{3}{2}}\left( 1-\dfrac{2Zr}{3a_0}+\dfrac{2(Zr)^2}{27a_0^2} \right)e^{-Zr/3a_0};\] 
\[ R_{31}=\dfrac{4\sqrt{2}}{3} \left( \dfrac{Z}{3a_0} \right)^{\frac{3}{2}}\left( \dfrac{Zr}{a_0} \right)\left(1- \dfrac{Zr}{6a_0} \right)e^{-Zr/3a_0};\,R_{32}=\dfrac{2\sqrt2}{27\sqrt{5}} \left( \dfrac{Z}{3a_0} \right)^{\frac{3}{2}}\left( \dfrac{Zr}{a_0} \right)^2e^{-Zr/3a_0};\]
\rule{\textwidth}{0.4pt}

\setstretch{1.35}

\section{\texorpdfstring{Энергии, молекулярные орбитали некоторых систем}{Энергии, молекулярные орбитали некоторых систем}}

\vspace{3mm}

\begin{center}
\begin{tabular}{ c c c }
\hline
 этилен & \begin{tabular}{@{}c@{}} аллильный\\радикал\end{tabular} & \begin{tabular}{@{}c@{}} циклопропенильный\\радикал\end{tabular} \\[1ex]
\hline\hline
\renewcommand{\arraystretch}{1.5} \begin{tabular}{@{}c@{}}  $E_2=\alpha-\beta$ \\ $\psi_2=\dfrac{1}{\sqrt{2}}(\phi_1-\phi_2)$ \end{tabular} & \renewcommand{\arraystretch}{1.5} \begin{tabular}{@{}c@{}}  $E_3=\alpha-\sqrt{2}\beta$ \\ $\psi_3=\dfrac{1}{{2}}(\phi_1-\sqrt{2}\phi_2+\phi_3)$ \end{tabular} & \renewcommand{\arraystretch}{1.5} \begin{tabular}{@{}c@{}}  $E_{2,\,3}=\alpha-\beta$ \\ $\psi_2=\dfrac{1}{\sqrt{6}}(2\phi_1-\phi_2-\phi_3)$ \\ $\psi_3=\dfrac{1}{\sqrt{2}}(\phi_2-\phi_3)$ \end{tabular} \\[7ex]
\hline
\renewcommand{\arraystretch}{1.5} \begin{tabular}{@{}c@{}}  $E_1=\alpha+\beta$ \\ $\psi_1=\dfrac{1}{\sqrt{2}}(\phi_1+\phi_2)$ \end{tabular} & \renewcommand{\arraystretch}{1.5} \begin{tabular}{@{}c@{}}  $E_2=\alpha$ \\ $\psi_2=\dfrac{1}{\sqrt{2}}(\phi_1-\phi_3)$ \end{tabular} & \renewcommand{\arraystretch}{1.5} \begin{tabular}{@{}c@{}}  $E_1=\alpha+2\beta$ \\ $\psi_1=\dfrac{1}{\sqrt{3}}(\phi_1+\phi_2+\phi_3)$ \end{tabular} \\[4ex]
\hline
 & \renewcommand{\arraystretch}{1.5} \begin{tabular}{@{}c@{}}  $E_1=\alpha+\sqrt2\beta$ \\ $\psi_1=\dfrac{1}{\sqrt{2}}(\phi_1+\sqrt{2}\phi_2+\phi_3)$ \end{tabular} & \\[4ex]
\hline
\end{tabular}
\end{center}

\setlength{\tabcolsep}{9pt} % Default value: 6pt
\renewcommand{\arraystretch}{1.25}
\vspace{5mm}
%\begin{tabular}{@{}c@{}} пентадиенильный\\радикал\end{tabular}
\begin{center}
\begin{tabular}{ c c }
\hline
 бутадиен & пентадиенильный радикал \\[0.5ex]
\hline\hline
\renewcommand{\arraystretch}{1.5} \begin{tabular}{@{}c@{}}  $E_4=\alpha-1,618\beta$ \\ $\psi_4=a\phi_1-b\phi_2+b\phi_3-a\phi_4$ \end{tabular} & \renewcommand{\arraystretch}{1.5} \begin{tabular}{@{}c@{}}  $E_5=\alpha-\sqrt{3}\beta$ \\ $\psi_5=\dfrac{1}{\sqrt{12}}(\phi_1-\sqrt{3}\phi_2+2\phi_3-\sqrt{3}\phi_4+\phi_5)$ \end{tabular} \\[4ex]
\hline
\renewcommand{\arraystretch}{1.5} \begin{tabular}{@{}c@{}}  $E_3=\alpha-0,618\beta$ \\ $\psi_3=b\phi_1-a\phi_2-a\phi_3+b\phi_4$ \end{tabular} & \renewcommand{\arraystretch}{1.5} \begin{tabular}{@{}c@{}}  $E_4=\alpha-\beta$ \\ $\psi_4=\dfrac{1}{{2}}(\phi_1-\phi_2+\phi_4-\phi_5)$ \end{tabular} \\[4ex]
\hline
\renewcommand{\arraystretch}{1.5} \begin{tabular}{@{}c@{}}  $E_2=\alpha+0,618\beta$ \\ $\psi_2=b\phi_1+a\phi_2-a\phi_3-b\phi_4$ \end{tabular} & \renewcommand{\arraystretch}{1.5} \begin{tabular}{@{}c@{}}  $E_3=\alpha$ \\ $\psi_3=\dfrac{1}{\sqrt{3}}(\phi_1-\phi_3+\phi_5)$ \end{tabular}\\[4ex]
\hline
\renewcommand{\arraystretch}{1.5} \begin{tabular}{@{}c@{}}  $E_4=\alpha+1,618\beta$ \\ $\psi_1=a\phi_1+b\phi_2+b\phi_3+a\phi_4$ \end{tabular} & \renewcommand{\arraystretch}{1.5} \begin{tabular}{@{}c@{}}  $E_5=\alpha+\beta$ \\ $\psi_2=\dfrac{1}{{2}}(\phi_1+\phi_2-\phi_4-\phi_5)$ \end{tabular} \\[4ex]
\hline
 $a = 0,372;\ $ $b=0,602$ & \renewcommand{\arraystretch}{1.5} \begin{tabular}{@{}c@{}}  $E_1=\alpha+\sqrt{3}\beta$ \\ $\psi_1=\dfrac{1}{\sqrt{12}}(\phi_1+\sqrt{3}\phi_2+2\phi_3+\sqrt{3}\phi_4+\phi_5)$ \end{tabular}  \\[4ex]
\hline
\end{tabular}
\end{center}

\renewcommand{\arraystretch}{1.5}
\begin{center}
\begin{tabular}{ c c }
\hline
циклобутадиен & бензол \\[0.5ex]
\hline\hline
\renewcommand{\arraystretch}{1.5} \begin{tabular}{@{}c@{}}  $E_4=\alpha-2\beta$ \\ $\psi_4=\dfrac{1}{{2}}(\phi_1-\phi_2+\phi_3-\phi_4)$ \end{tabular}  & \renewcommand{\arraystretch}{1.5} \begin{tabular}{@{}c@{}}  $E_6=\alpha-2\beta$ \\ $\psi_6=\dfrac{1}{\sqrt{6}}(\phi_1-\phi_2+\phi_3-\phi_4+\phi_5-\phi_6)$ \end{tabular} \\[4ex]
\hline
\renewcommand{\arraystretch}{1.5} \begin{tabular}{@{}c@{}}  $E_{2,\,3}=\alpha$ \\ $\psi_2=\dfrac{1}{\sqrt{2}}(\phi_1-\phi_3)$ \\ $\psi_3=\dfrac{1}{\sqrt{2}}(\phi_2-\phi_4)$ \end{tabular} &  \renewcommand{\arraystretch}{1.5} \begin{tabular}{@{}c@{}c@{}}  $E_{4,\,5}=\alpha-\beta$ \\ $\psi_4=\dfrac{1}{\sqrt{12}}(2\phi_6-\phi_1-\phi_5+2\phi_3-\phi_2-\phi_4)$ \\ $\psi_5=\dfrac{1}{{2}}(\phi_1-\phi_2+\phi_4-\phi_5)$  \end{tabular} \\[7ex]
\hline
\renewcommand{\arraystretch}{1.5} \begin{tabular}{@{}c@{}}  $E_1=\alpha+2\beta$ \\ $\psi_1=\dfrac{1}{{2}}(\phi_1+\phi_2+\phi_3+\phi_4)$ \end{tabular} & \renewcommand{\arraystretch}{1.5} \begin{tabular}{@{}c@{}c@{}}  $E_{2,\,3}=\alpha+\beta$ \\ $\psi_2=\dfrac{1}{\sqrt{12}}(2\phi_6+\phi_1+\phi_5-2\phi_3-\phi_2-\phi_4)$ \\ $\psi_3=\dfrac{1}{{2}}(\phi_1+\phi_2-\phi_4-\phi_5)$ \end{tabular} \\[7ex]
\hline
& \renewcommand{\arraystretch}{1.5} \begin{tabular}{@{}c@{}}  $E_1=\alpha+2\beta$ \\ $\psi_1=\dfrac{1}{\sqrt{6}}(\phi_1+\phi_2+\phi_3+\phi_4+\phi_5+\phi_6)$ \end{tabular} \\[4ex]
\hline
\end{tabular}
\end{center}

\vspace{1mm}
\setstretch{1.35}

\section{Общие решения некоторых специальных систем}
\subsection{Циклические системы}
\vspace{1mm}
\begin{tabular}{ c c }
$E_k=\alpha +2\beta\cos{\dfrac{2\pi k}{n}}$ & $k=1,\,2,\,\ldots\,n $ \\[2ex]
$\psi_k=\sqrt{\dfrac{1}{n}}\,\mathlarger{\mathlarger{\sum}}\limits_{p=1}^{n} \cos\left({\dfrac{2\pi p k}{n}\phi_p}\right)$ &  \begin{tabular}{ c c }  $k=0$ & $n=2i+1$ \\[0ex] $ k=0,\,\dfrac{n}{2}$ & $n=2i$ \end{tabular} \\[5ex]
$\psi_k=\sqrt{\dfrac{2}{n}}\,\mathlarger{\mathlarger{\sum}}\limits_{p=1}^{n} \cos\left({\dfrac{2\pi p k}{n}\phi_p}\right)$ &  \begin{tabular}{ c c }  $k=1,\,2,\,\ldots \dfrac{n-1}{2}$ & $n=2i+1$ \\[1ex] $k=1,\,2,\,\ldots \dfrac{n-2}{2}$ & $n=2i$ \end{tabular} \\[5ex]
$\xi_k=\sqrt{\dfrac{2}{n}}\,\mathlarger{\mathlarger{\sum}}\limits_{p=1}^{n} \sin\left({\dfrac{2\pi p k}{n}\phi_p}\right)$ &  \begin{tabular}{ c c }  $k=1,\,2,\,\ldots \dfrac{n-1}{2}$ & $n=2i+1$ \\[1ex] $k=1,\,2,\,\ldots \dfrac{n-2}{2}$ & $n=2i$ \end{tabular} \\[5ex]
\end{tabular}

\subsection{Линейные системы}
\vspace{1mm}
\begin{tabular}{ c c }
$E_k=\alpha +2\beta\cos{\dfrac{\pi k}{n+1}}$ & $k=1,\,2,\,\ldots\,n $ \\[3ex]
$\psi_k=\sqrt{\dfrac{2}{n+1}}\,\mathlarger{\mathlarger{\sum}}\limits_{p=1}^{n} \sin\left({\dfrac{\pi p k}{n+1}\phi_p}\right)$ & $k=1,\,2,\,\ldots\,n $  \\[5ex]
\end{tabular}

\subsection{Альтернантные системы}
\vspace{1mm}
1) Уровни энергии альтернантных систем расположены симметричными относительно нулевого уровня парами: $E=\alpha\,\pm k\beta$. Соответствующие молекулярные орбитали
отличаются лишь заменой знаков у одного из наборов коэффициентов $\vec{C}$ или $\vec{C^*}$. Все связывающие молекулярные орбитали заняты, все разрыхляющие молекулярные орбитали свободны. Неспаренный электрон/электроны находятся на уровне $E=\alpha$. Число несвязывающих молекулярных орбиталей $Z$ не меньше $N-2T$, где $N$ – число атомов углерода в альтернантной системе, $T$ – максимальное число двойных связей в резонансных структурах, то есть $Z \geq N-2T$.

2) $\pi$-заряд на всех центрах альтернантой системы равен 1.

3) Для нечетных альтернантных систем все немеченые центры имеют коэффициенты равные 0 в молекулярной орбитали неспаренного электрона. Сумма коэффициентов меченых центров вокруг каждого немеченого центра равна 0.


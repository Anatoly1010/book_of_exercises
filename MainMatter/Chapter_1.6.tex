\setmainfont{Noto Serif}
\setsansfont{Noto Sans}
\setmonofont{Noto Sans Mono}
\setstretch{1.35}
\hyphenation{ма-те-ма-ти-ка вос-ста-нав-ли-вать}


\section{Электронное строение двухатомных молекул}

1С. С помощью теории МО ЛКАО рассмотреть электронное строение молекулы фтора. Определите основной терм, а также термы катион-, анион-радикала молекулы фтора и термы первого возбужденного состояния. Укажите также несколько разрешенных электрических дипольных переходов в данной молекуле.
\par
2. Проверьте, что дипольный момент молекулы водорода в основном и возбужденном состоянии равен нулю, в то время как дипольный момент одноэлектронного перехода отличен от нуля.
\par
3С. Молекула угарного газа является хорошим лигандом в координационной химии. Используя теорию МО ЛКАО объясните с чем это связано. Каким атомом координируется угарный газ в комплексах? Как это можно объяснить исходя из его электронного строения? Энергия атомных орбиталей углерода равна $E_{2s}$ = $-$19,4 эВ, $E_{2p}$ = $-$10,7 эВ. Энергия атомных орбиталей кислорода равна $E_{2s}$~= $-$32,4 эВ, $E_{2p}$ = $-$15,9 эВ. Точный расчет (RHF/3-21G*) показывает, что вклад различных атомных орбиталей в третью по энергии $\sigma$ молекулярную орбиталь равен, соответственно, $\text{C}_{2p_z}$ $\approx$ 32\%, $\text{C}_{2s}$ $\approx$ 56\%, $\text{O}_{2p_z}$ $\approx$ 11\%.
\par
4С. С помощью теории МО ЛКАО показать, что  $\text{He}_2$ является неустойчивой молекулой, а $\text{He}_2^{+}$ – устойчивой. Определите порядок связи в указанных двух молекулах.
\par
5. Определите основной терм $\text{OH}^-$ и $\text{OH}$.
\par
6К. Определите основной терм молекулы $\text{NO}$. В результате значительного спин-орбитального взаимодействия $\text{NO}$ при низкой температуре практически не~проявляет магнитных свойств. Объясните этот факт, считая, что при учете спин-орбитального взаимодействия состояния молекулы $^{2S+1}\Lambda$ расщепляются по величине полной проекции на ось молекулы $\Omega = \Lambda + M_s$, которая пробегает ряд значений от $\Lambda + S$ до $\Lambda - S$ через единицу. Почему магнитные свойства появляются при повышении температуры?
\par
7. Найти основной терм молекулы $\text{MoC}$. Можно ли получить основное электронное состояние $\text{MoC}$ при сведении атомов $\text{Mo}$ и $\text{C}$ в основных состояниях. Энергия атомных орбиталей углерода равна $E_{2s}$ = $-$19,4 эВ, $E_{2p}$ = $-$10,7 эВ. Энергия атомных орбиталей молибдена равна $E_{5s}$ = $-$7,4 эВ, $E_{5d}$ = $-$9,1 эВ.
\par
8. Найти основной терм линейной молекулы $\text{C}_3$. Что изменится при замене одного или двух атомов углерода на азот? Рассмотреть все возможные варианты замены атомов углерода. 
\par
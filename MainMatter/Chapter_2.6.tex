\setmainfont{Noto Serif}
\setsansfont{Noto Sans}
\setmonofont{Noto Sans Mono}
\setstretch{1.35}

\section{Вращательная спектроскопия и ее производные}
1С. Для двухатомной молекулы XY в состоянии с колебательным квантовым числом $\nu$ = 1 межъядерное расстояние увеличено на 1\% по сравнению с расстоянием для $\nu$ = 0. Какая линия R-ветви в колебательно-вращательном спектре будет соответствовать максимальной энергии (предел канта).
\par
2С. Вращательные переходы $J + 1$ $\leftarrow$ $J$ в двух линейных изотопологах $^{16}\text{O}^{12}\text{C}^{32}\text{S}$ и~$^{16}\text{O}^{12}\text{C}^{34}\text{S}$ соответствуют частотам, приведенным в~единицах ГГц для некоторых $J$ в таблице ниже.
\begin{center}
\begin{tabular}{ c c c c c }

 J & 1 & 2 & 3 & 4 \\ 
 \hline
 $^{16}$O$^{12}$C$^{32}$S & 24,32592 & 36,48882 & 48,65164 & 60,81408 \\  
 $^{16}$O$^{12}$C$^{34}$S & 23,73223 & $-$ & 47,4620 & $-$    
\end{tabular}
\end{center}
Определите величину вращательной постоянной и момент инерции для изотополога $^{16}\text{O}^{12}\text{C}^{32}\text{S}$, а также длины связей $\text{O}-\text{C}$ и $\text{S}-\text{C}$, считая что для данных изотопологов оно одинаковое.
\par
3С. Вращательная постоянная молекулы $^1$H$^{35}$Cl равна 10,4400 см$^{-1}$ для основного колебательного состояния и 10,1366 см$^{-1}$ для первого возбужденного. Нарисуйте вращательно-колебательный спектр данной молекулы в окрестности фундаментальной частоты 2990,95 см$^{-1}$.
\par
4С. Покажите, что вращательные энергетические уровни плоской квадратной молекулы XY$_4$ могут быть выражены используя только вращательную постоянную $B$.
\par
5. Трехатомная молекула имеет строение X$_2$Y. Ее вращательный спектр имеет линии поглощения с частотой 20 ГГц, 40 ГГц, 60 ГГц и т.д. Какая из следующих структур соответствует данной молекуле: X$-$X$-$Y (линейная), X$-$Y$-$X (линейная), X$-$Y$-$X (изогнутая), X$-$X$-$Y (изогнутая)?
\par
6С. Классифицируйте ядерные спиновые изомеры линейной молекулы $\text{BeH}_2$. Ядерный спин единственного природного изотопа равен $I$($^9\text{Be}$) = 3/2.
\par
7С. Постройте вращательно-колебательной спектр молекулы углекислого газа.
\par
8С. Постройте вращательно-колебательный спектр молекулы ацетилена, считая что ее основной терм $^1\Sigma^+_g$. Какие изменения произойдут в спектре если все углеродные атомы $^{12}\text{C}$ заменить на $^{13}\text{C}$?
\par
9С. Определите процент линейных молекул $\text{Cl}-\text{O}-\text{Cl}$, имеющих только четные вращательные уровни. $I$($^{35}\text{Cl}$) = $I$($^{37}\text{Cl}$) = 3/2. Считайте, что природное содержание изотопа $^{35}\text{Cl}$ в три раза больше, чем содержание изотопа $^{37}\text{Cl}$.
\par
10. При каком ядерном спине в чисто вращательном спектре комбинационного рассеяния двухатомной гомоядерной молекулы интенсивность первой и~второй линии будет одинакова?
\par
11К. В лаборатории физиологии фантастических тварей проводились исследования источника топлива огненного дыхания Змея Горыныча. Физико-хими-ческий анализ несгораемых остатков показал, что топливо представляет собой двухатомную молекулу T с приведенной массой 1,615$\cdot$10$^{-27}$ кг. Микроволновый спектр соединения T содержит набор эквидистантных линий с расстоянием между линиями 22,04 см$^{-1}$. Используя приближение жесткого ротора определите длину связи в молекуле T. Что это может быть за молекула? Примечание: 1 а. е. м. = 1,6605$\cdot$10$^{-27}$ кг, $\hbar$ = 1,05$\cdot$10$^{-34}$ Дж$\cdot$с.
\par
12К. Из термодинамических соображений следует, что моногалогениды меди CuX должны существовать в газовой фазе в основном в виде полимеров, что подтверждается экспериментально. Эта проблема может быть решена путем пропускания газообразного галогена над медью, нагретой до 1100 К. Так, в~газообразном $^{63}\text{Cu}^{81}\text{Br}$ были зарегистрированы переходы между вращательными уровнями с квантовыми числами 13 $\rightarrow$ 14, 14 $\rightarrow$ 15 и 15 $\rightarrow$ 16 на частотах 84421,34; 90449,25 и 96476,72 МГц, соответственно. Определите вращательную постоянную и длину связи для $^{63}\text{Cu}^{81}\text{Br}$. Примечание: 1 а. е. м. = 1,6605$\cdot$10$^{-27}$ кг, $\hbar$ = 1,05$\cdot$10$^{-34}$ Дж$\cdot$с.
\par
13К. Квадрат электрического дипольного момента перехода между двумя вращательными уровнями зависит от вращательного квантового числа $J$ следующим образом: $|d_{J+1,\,J}|^2 \sim \frac{J+1}{2J+1}$. Предскажите вид чисто вращательного спектра молекулы $^{1}\text{H}^{35}\text{Cl}$ при температуре 300 К без учета и с учетом приведенного фактора. В частности, в каждом случае определите значение $J$, соответствующее линии поглощения с наибольшей относительной интенсивностью. Вращательная постоянная $^{1}\text{H}^{35}\text{Cl}$ равна 10,6 см$^{-1}$.
\par